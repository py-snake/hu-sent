\documentclass[12pt]{article}
\usepackage[utf8]{inputenc}
\usepackage[T1]{fontenc}
\usepackage[magyar]{babel}
\usepackage{graphicx}
\usepackage{hyperref}
\usepackage{listings}
\usepackage{xcolor}
\usepackage{float}

\title{Magyar nyelvű Szentiment Analízis Projekt Terv}
\author{Név}
\date{\today}

\definecolor{codegreen}{rgb}{0,0.6,0}
\definecolor{codegray}{rgb}{0.5,0.5,0.5}
\definecolor{codepurple}{rgb}{0.58,0,0.82}
\definecolor{backcolour}{rgb}{0.95,0.95,0.92}

\lstdefinestyle{mystyle}{
    backgroundcolor=\color{backcolour},   
    commentstyle=\color{codegreen},
    keywordstyle=\color{magenta},
    numberstyle=\tiny\color{codegray},
    stringstyle=\color{codepurple},
    basicstyle=\ttfamily\footnotesize,
    breakatwhitespace=false,         
    breaklines=true,                 
    captionpos=b,                    
    keepspaces=true,                 
    numbers=left,                    
    numbersep=5pt,                  
    showspaces=false,                
    showstringspaces=false,
    showtabs=false,                  
    tabsize=2
}

\lstset{style=mystyle}

\begin{document}

\maketitle

\section{Projekt Áttekintés}
Ez a projekt célja egy magyar nyelvű szentiment analízis modell fejlesztése Pythonban, amely a HuSST adatkészletet használja. A modell felé elvárás, hogy képes legyen szövegeket negatív, semleges és pozitív kategóriákba sorolni.

\section{Módszertan}
A cél megvalósításához a huBERT betanított neurális hálót fogom felhasználni alapmodellként. Az előre betanított neurális háló nagyon jó kiindulási alapként szolgál, mivel magyar nyelvű adatokon tanították tehát általános magyar nyelvtudással rendelkezik. Képes a szövegek értelmezésére és feldolgozására, viszont általánosságban elmondható, hogy ezeket az alapmodelleket további tanítással kell kiegészíteni ha specifikusan egy bizonyos célra szeretnénk használni a tudását.

Jelen feladatban a HuSST adathalmazzal fogok további tanítást végezni a modellen. A HuSST mint korábban említsre került, magyar nyelvű kijelentéseket tartalmaz és az azokhoz tartozó címkét. A címke lehet negatív, semleges, vagy pozitív. Ezek alapján kerül besorolásra az adott szöveg. 


\section{Dataset}
A bevezetőben ismertetett két forrást fogom használni a projekt megvalósításához.
\begin{itemize}
    \item huBERT base model (Hungarian Universal Bidirectional Encoder Representations from Transformers)
    \item HuSST dataset (Hungarian Stanford Sentiment Treebank)
\end{itemize}

\section{Implementáció}
A modell Pythonban készül a következő könyvtárakkal:

\begin{itemize}
    \item torch
    \item sklearn
    \item transformers
    \item datasets
    \item tqdm
\end{itemize}

\section{Források}

A dokumentumot az alább felsorolt források segítségével készítettem el.

\begin{thebibliography}{9}

\bibitem{hubert-hf}
SZTAKI-HLT. (2022). \textit{hubert-base-cc}. Hugging Face. \\
\url{https://huggingface.co/SZTAKI-HLT/hubert-base-cc}

\bibitem{husset}
NYTK. (2022). \textit{HuSST Dataset}. Hugging Face. \\
\url{https://huggingface.co/datasets/NYTK/HuSST}

\bibitem{hubert-official}
SZTAKI-HLT. (2022). \textit{huBERT - Hungarian BERT Model}. BME-HLT. \\
\url{https://hlt.bme.hu/hu/resources/hubert}

\bibitem{awesome-nlp}
Orosz György. (2023). \textit{Awesome Hungarian NLP Resources}. GitBook. \\
\url{https://oroszgy.gitbook.io/awesome-hungarian-nlp-resources}

\bibitem{awesome-github}
Orosz György. (2023). \textit{Awesome Hungarian NLP}. GitHub. \\
\url{https://github.com/oroszgy/awesome-hungarian-nlp}

\bibitem{hubert-paper}
Laki László J., Yang Zijian Győző. (2022). \textit{huBERT - Hungarian BERT}. Acta Universitatis Óbuda. \\
\url{https://acta.uni-obuda.hu/Laki_Yang_134.pdf}

\end{thebibliography}


\end{document}

